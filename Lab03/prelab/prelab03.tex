\documentclass[11pt]{article}
\usepackage[margin=1in]{geometry}
\usepackage{graphicx}
\usepackage{float}
\usepackage{booktabs}
\usepackage{siunitx}
\usepackage{amsmath}
\usepackage{amssymb}
\usepackage{pgfplots}
\pgfplotsset{compat=1.18}

\title{ENGR 200: Pre-Lab 3}
\author{Sean Balbale}
\date{February 3, 2026}
\setlength{\parindent}{0in}
\setlength{\parskip}{\baselineskip}

\begin{document}

\begin{titlepage}
  \begin{center}
    \vspace*{1in}

    \Huge
    \textbf{Pre-Lab 3}

    \LARGE
    Uncertainty Analysis

    \vspace{3in}

    \textbf{Student Name:} Sean Balbale
    \\ \textbf{Course:} ENGR 200
    \\ \textbf{Instructor:} Prof. R. Gao
    \\ \textbf{Date:} February 6, 2026

    \vfill

  \end{center}
\end{titlepage}

\newpage

% ----------------------------------------------------------------------
% PROBLEM 1
% ----------------------------------------------------------------------
\section*{Problem 1: Reporting Significant Figures}

\textbf{Prompt:} You are reading the reported results from an experiment, which lists a value for the length of a part as $53.087 \pm 1.2$ \si{\centi\meter}. Comment on what is wrong with this reported value and what should be given instead.

\textbf{Analysis:}
The reported value ($53.087$) claims precision to the thousandths place, while the uncertainty ($1.2$) indicates that the measurement is only reliable to the tenths (or arguably the ones) place. Reporting digits beyond the magnitude of the uncertainty implies a level of precision that does not exist, vitiating the statistical validity of the result.

\textbf{Correction:}
The uncertainty determines the last significant digit of the reported value.
\begin{enumerate}
  \item Round the uncertainty to one significant figure: $\pm 1$ \si{\centi\meter}.
  \item Round the measurement to the corresponding decimal place (the ones place): $53$ \si{\centi\meter}.
\end{enumerate}

\textbf{Corrected Result:}
\[ L = 53 \pm 1 \text{ \si{\centi\meter}} \]

% ----------------------------------------------------------------------
% PROBLEM 2
% ----------------------------------------------------------------------
\section*{Problem 2: Calculating Instrumentation Uncertainty}

\textbf{Prompt:} You are measuring the length of the side of a surface with digital calipers. The calipers have a reported accuracy of 0.1\% of the measurement with a screen resolution of 0.1 mm. What is the uncertainty of a measurement of 10.00 cm?

\textbf{Given:}
\begin{itemize}
  \item Measurement ($L$) = \SI{10.00}{\centi\meter}
  \item Accuracy ($u_{acc}$) = $0.1\%$ of reading
  \item Resolution ($u_{res}$) = \SI{0.1}{\milli\meter} = \SI{0.01}{\centi\meter}
\end{itemize}

\textbf{Calculation:}
The total uncertainty $u_L$ is the Root Sum Square (RSS) of the systematic accuracy uncertainty and the resolution uncertainty.

1. Calculate accuracy component:
\[ u_{acc} = 0.001 \times 10.00 \text{ \si{\centi\meter}} = 0.01 \text{ \si{\centi\meter}} \]

2. Calculate resolution component:
\[ u_{res} = 0.01 \text{ \si{\centi\meter}} \]

3. Calculate total uncertainty:
\[ u_L = \sqrt{u_{acc}^2 + u_{res}^2} = \sqrt{(0.01)^2 + (0.01)^2} = \sqrt{0.0002} \approx 0.0141 \text{ \si{\centi\meter}} \]

\textbf{Answer:}
The uncertainty is $\pm 0.014$ \si{\centi\meter}.

% ----------------------------------------------------------------------
% PROBLEM 3
% ----------------------------------------------------------------------
\section*{Problem 3: Propagation of Uncertainty}

\textbf{Prompt:} Using the calipers from Problem 2, you are estimating the area of a rectangular plate. You measure the base to have a length of $l_b = 20.26$ cm and the height to have a height of $l_h = 9.60$ cm. How would you report the area of your rectangular plate, including your uncertainty?

\textbf{1. Calculate the Area ($A$):}
\[ A = l_b \times l_h = 20.26 \text{ \si{\centi\meter}} \times 9.60 \text{ \si{\centi\meter}} = 194.496 \text{ \si{\square\centi\meter}} \]

\textbf{2. Calculate Component Uncertainties:}
Using the method from Problem 2 ($u = \sqrt{u_{acc}^2 + u_{res}^2}$):

\textit{For Base ($l_b = 20.26$ \si{\centi\meter}):}
\[ u_{acc,b} = 0.001 \times 20.26 = 0.02026 \text{ \si{\centi\meter}} \]
\[ u_b = \sqrt{(0.02026)^2 + (0.01)^2} \approx 0.0226 \text{ \si{\centi\meter}} \]

\textit{For Height ($l_h = 9.60$ \si{\centi\meter}):}
\[ u_{acc,h} = 0.001 \times 9.60 = 0.0096 \text{ \si{\centi\meter}} \]
\[ u_h = \sqrt{(0.0096)^2 + (0.01)^2} \approx 0.0139 \text{ \si{\centi\meter}} \]

\textbf{3. Propagate Uncertainty to Area ($u_A$):}
For a product $A = l_b \cdot l_h$, the relative uncertainty is calculated using RSS:
\[ \frac{u_A}{A} = \sqrt{\left(\frac{u_b}{l_b}\right)^2 + \left(\frac{u_h}{l_h}\right)^2} \]

Substituting values:
\[ \frac{u_b}{l_b} = \frac{0.0226}{20.26} \approx 0.001115 \]
\[ \frac{u_h}{l_h} = \frac{0.0139}{9.60} \approx 0.001448 \]

\[ \frac{u_A}{A} = \sqrt{(0.001115)^2 + (0.001448)^2} \approx \sqrt{1.24 \times 10^{-6} + 2.10 \times 10^{-6}} \approx 0.00183 \]

\[ u_A = A \times 0.00183 = 194.496 \times 0.00183 \approx 0.356 \text{ \si{\square\centi\meter}} \]

Rounding the uncertainty to one significant figure ($0.4$) and the area to the corresponding decimal place (tenths):

\[ A = 194.5 \pm 0.4 \text{ \si{\square\centi\meter}} \]

\end{document}
